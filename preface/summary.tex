\cleardoublepage
\phantomsection
\pdfbookmark{Sommario}{Sommario}
\begingroup
\let\clearpage\relax
\let\cleardoublepage\relax
\let\cleardoublepage\relax

\chapter*{Sommario}

Il presente documento descrive il lavoro svolto durante il periodo di stage, della durata di circa trecento ore, dal laureando Francesco Lapenna presso il Dipartimento di Matematica dell'Università degli Studi di Padova.
Gli obbiettivi da raggiungere erano molteplici.\\
In primo luogo era richiesto la configurazione di due schede BeagleBone Black per consentire la comunicazione tramite luce visibile.
In secondo luogo era richiesta l'implementazione di un modulo di autenticazione sulla piattaforma OpenVLC seguendo lo standard IEEE 802.15.7.
Tale modulo permette di autenticare rapidamente i messaggi, con basso impatto sulla velocità di trasmissione e di individuare pacchetti non validi già al primo stadio della comunicazione.
Infine era richiesto di testare il sistema in diverse condizioni ambientali.

%\vfill

%\selectlanguage{english}
%\pdfbookmark{Abstract}{Abstract}
%\chapter*{Abstract}

%\selectlanguage{italian}

\endgroup

\vfill
