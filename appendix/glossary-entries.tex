% ================================================================= %
% Acronyms
% ================================================================= %
\newacronym[description={\glslink{apig}{Application Program Interface}}]
    {api}{API}{Application Program Interface}

\newacronym[description={\glslink{umlg}{Unified Modeling Language}}]
    {uml}{UML}{Unified Modeling Language}

\newacronym[description={\glslink{vlcg}{Visible Light Communication}}]
    {vlc}{VLC}{Visible Light Communication}

\newacronym[description={\glslink{itsg}{Intelligent Transportation System}}]
    {its}{ITS}{Intelligent Transportation System}

\newacronym[description={\glslink{iotg}{Internet of Things}}]
    {iot}{IoT}{Internet of Things}

\newacronym[description={\glslink{ubicompg}{Ubiquitous Computing}}]
    {ubicomp}{ubicomp}{Ubiquitous Computing}

\newacronym[description={\glslink{ipsg}{Indoor Positioning System}}]
    {ips}{IPS}{Indoor Positioning System}

\newacronym[description={\glslink{ieeeg}{Institute of Electrical and Electronics Engineers}}]
    {ieee}{IEEE}{Institute of Electrical and Electronics Engineers}

\newacronym[description={\glslink{osig}{Open Systems Interconnection Model}}]
    {osi}{OSI}{Open Systems Interconnection Model}

\newacronym[description={\glslink{phyg}{Physical Layer}}]
    {phy}{PHY}{Physical Layer}

\newacronym[description={\glslink{macg}{Medium Access Control Layer}}]
    {mac}{MAC}{Medium Access Control Layer}

\newacronym[description={\glslink{otpg}{One Time Password}}]
    {otp}{OTP}{One Time Password}

\newacronym[description={\glslink{berg}{Bit Error Ratio}}]
    {ber}{BER}{Bit Error Ratio}

\newacronym[description={\glslink{prrg}{Packet Reception Rate}}]
    {prr}{PRR}{Packet Reception Rate}

\newacronym[description={\glslink{plrg}{Packet Loss Rate}}]
    {plr}{PLR}{Packet Loss Rate}

\newacronym[description={\glslink{prug}{Programmable Real-Time Unit}}]
    {pru}{PRU}{Programmable Real-Time Unit}

\newacronym[description={\glslink{ookg}{On–off keying}}]
    {ook}{OOK}{On–off keying}


% ================================================================= %
% Glossary entries
% ================================================================= %
\newglossaryentry{apig} {
    name=\glslink{api}{API},
    text=Application Program Interface,
    sort=api,
    description={in informatica con il termine \emph{Application Programming Interface API} (ing. interfaccia di programmazione di un'applicazione) si indica ogni insieme di procedure disponibili al programmatore, di solito raggruppate a formare un set di strumenti specifici per l'espletamento di un determinato compito all'interno di un certo programma. La finalità è ottenere un'astrazione, di solito tra l'hardware e il programmatore o tra software a basso e quello ad alto livello semplificando così il lavoro di programmazione}
}

\newglossaryentry{umlg} {
    name=\glslink{uml}{UML},
    text=UML,
    sort=uml,
    description={in ingegneria del software \emph{UML, Unified Modeling Language} (ing. linguaggio di modellazione unificato) è un linguaggio di modellazione e specifica basato sul paradigma object-oriented. L'\emph{UML} svolge un'importantissima funzione di ``lingua franca'' nella comunità della progettazione e programmazione a oggetti. Gran parte della letteratura di settore usa tale linguaggio per descrivere soluzioni analitiche e progettuali in modo sintetico e comprensibile a un vasto pubblico}
}

\newglossaryentry{vlcg} {
    name=\glslink{vlc}{VLC},
    text=Visible Light Communication,
    sort=vlc,
    description={nelle telecomunicazioni con il termine \emph{Visible Light Communication VLC} (ing. comunicazione attraverso luce visibile) si indica l'uso di luce visibile (luce con una frequenza di 400-800 THz/wavelength di 780-375 nm) come mezzo di trasmissione. VLC è un sottoinsieme delle tecnologie wireless di comunicazione ottica}
}

\newglossaryentry{itsg} {
    name=\glslink{its}{ITS},
    text=Intelligent Transportation System,
    sort=its,
    description={con il termine \emph{Intelligent Transportation System ITS} (ing. sistemi di trasporto intelligenti) si intende: l'integrazione delle conoscenze nel campo delle telecomunicazioni, elettronica, informatica - in breve, la “telematica” - con l'ingegneria dei trasporti, per la pianificazione, progettazione, esercizio, manutenzione e gestione dei sistemi di trasporto}
}

\newglossaryentry{iotg} {
    name=\glslink{iot}{IoT},
    text=Internet of Things,
    sort=iot,
    description={l'\emph{Internet of Things IoT} (ing. Internet delle cose (IdCa)) è un neologismo utilizzato nel mondo delle telecomunicazioni e dell'informatica che fa riferimento all'estensione di internet al mondo degli oggetti e dei luoghi concreti, che acquisiscono una propria identità digitale in modo da poter comunicare con altri oggetti nella rete stessa}
}

\newglossaryentry{ubicompg} {
    name=\glslink{ubicomp}{ubicomp},
    text=Ubiquitous Computing,
    sort=ubicomp,
    description={l'\emph{Ubiquitous Computing (o "ubicomp")} (ing. Computazione ubiqua) è un paradigma di calcolo in cui i computer sono integrati in oggetti di uso quotidiano e nell'ambiente circostante}
}

\newglossaryentry{ipsg} {
    name=\glslink{ips}{IPS},
    text=Indoor Positioning System,
    sort=ips,
    description={l'\emph{Indoor Positioning System IPS} (ing. Sistema di Posizionamento Indoor) prevede l'uso di dispositivi di rete al fine di localizzare persone o oggetti in luoghi in cui il GPS mancano di precisione o non sono disponibili, come ad esempio all'interno di edifici e strutture sotterranee}
}

\newglossaryentry{ieeeg} {
    name=\glslink{ieee}{IEEE},
    text=Institute of Electrical and Electronics Engineers,
    sort=ieee,
    description={L'\emph{Institute of Electrical and Electronics Engineers IEEE} è un'associazione internazionale di scienziati professionisti con l'obiettivo della promozione delle scienze tecnologiche}
}

\newglossaryentry{osig} {
    name=\glslink{osi}{OSI},
    text=Open Systems Interconnection Model,
    sort=osi,
    description={L'\emph{Open Systems Interconnection Model} (ing. Modello OSI), sviluppato dalla International Organization for Standardization (ISO), è lo standard architetturale di riferimento per le reti di calcolatori}
}

\newglossaryentry{phyg} {
    name=\glslink{phy}{PHY},
    text=Physical Layer,
    sort=phy,
    description={Il'\emph{Physical Layer} (ing. Livello fisico), è il primo livello del modello OSI. Si occupa della trasmissione e ricezione dei dati grezzi su un canale di comunicazione fisico}
}

\newglossaryentry{macg} {
    name=\glslink{mac}{MAC},
    text=Medium Access Control Layer,
    sort=mac,
    description={Il'\emph{Medium Access Control Layer}, è il livello che controlla l'hardware responsabile dell'interazione con il mezzo di trasmissione. Insieme al logical link control layer (LLC) costituisce il data link layer del modello OSI}
}

\newglossaryentry{otpg} {
    name=\glslink{otp}{OTP},
    text=One Time Password,
    sort=otp,
    description={Una \emph{One Time Password} (ing. password monouso), è una password valida per una singola sessione o transazione, e spesso solo per un breve periodo di tempo} 
}

\newglossaryentry{berg} {
    name=\glslink{ber}{BER},
    text=Bit Error Ratio,
    sort=ber,
    description={Il \emph{Bit Error Ratio} (ing. tasso di errore sui bit), è il numero di errore sui bit diviso per il numero totale dei bit trasmessi durante un dato intervallo di tempo}
}

\newglossaryentry{prrg} {
    name=\glslink{prr}{PRR},
    text=Packet Reception Rate,
    sort=prr,
    description={La \emph{Packet Reception Rate} (ing. tasso di ricezione dei pacchetti), è definita come la percentuale di pacchetti correttamente ricevuti rispetto al totale dei pacchetti trasmessi}
}

\newglossaryentry{plrg} {
    name=\glslink{plr}{PLR},
    text=Packet Loss Rate,
    sort=plr,
    description={La \emph{Packet Loss Rate} (ing. tasso di perdita dei pacchetti), è definita come la percentuale di pacchetti trasmessi che non vengono correttamente ricevuti dal destinatario}
}

\newglossaryentry{prug} {
    name=\glslink{pru}{PRU},
    text=Programmable Real-Time Unit,
    sort=pru,
    description={Le \emph{Programmable Real-time Unit}, sono microcontroller integrati all'interno di alcuni System-on-Chip (SoC), progettati per eseguire compiti a bassa latenza in tempo reale, in parallelo con il processore principale. Ogni PRU è una piccola CPU indipendente, programmabile direttamente, e dotata di accesso diretto alla memoria, a periferiche e a linee di I/O, il che consente di ottenere un controllo preciso sul timing delle operazioni}
}

\newglossaryentry{ookg} {
    name=\glslink{ook}{OOK},
    text=On–off keying,
    sort=ook,
    description={\emph{On–off keying OOK} denota la più semplice forma di modulazione ASK (amplitude-shift keying), la quale rappresenta i dati digitali come presenza o assenza di una portante}
}