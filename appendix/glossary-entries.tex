% ----------------------------------------------------------------- %
% Acronyms
% ----------------------------------------------------------------- %
\newacronym[description={\glslink{apig}{Application Program Interface}}]
    {api}{API}{Application Program Interface}

\newacronym[description={\glslink{umlg}{Unified Modeling Language}}]
    {uml}{UML}{Unified Modeling Language}

\newacronym[description={\glslink{vlcg}{Visible Light Communication}}]
    {vlc}{VLC}{Visible Light Communication}

\newacronym[description={\glslink{itsg}{Intelligent Transportation System}}]
    {its}{ITS}{Intelligent Transportation System}

\newacronym[description={\glslink{iotg}{Internet of Things}}]
    {iot}{IoT}{Internet of Things}



% ----------------------------------------------------------------- %
% Glossary entries
% ----------------------------------------------------------------- %
\newglossaryentry{apig} {
    name=\glslink{api}{API},
    text=Application Program Interface,
    sort=api,
    description={in informatica con il termine \emph{Application Programming Interface API} (ing. interfaccia di programmazione di un'applicazione) si indica ogni insieme di procedure disponibili al programmatore, di solito raggruppate a formare un set di strumenti specifici per l'espletamento di un determinato compito all'interno di un certo programma. La finalità è ottenere un'astrazione, di solito tra l'hardware e il programmatore o tra software a basso e quello ad alto livello semplificando così il lavoro di programmazione}
}

\newglossaryentry{umlg} {
    name=\glslink{uml}{UML},
    text=UML,
    sort=uml,
    description={in ingegneria del software \emph{UML, Unified Modeling Language} (ing. linguaggio di modellazione unificato) è un linguaggio di modellazione e specifica basato sul paradigma object-oriented. L'\emph{UML} svolge un'importantissima funzione di ``lingua franca'' nella comunità della progettazione e programmazione a oggetti. Gran parte della letteratura di settore usa tale linguaggio per descrivere soluzioni analitiche e progettuali in modo sintetico e comprensibile a un vasto pubblico}
}

\newglossaryentry{vlcg} {
    name=\glslink{vlc}{VLC},
    text=Visible Light Communication,
    sort=vlc,
    description={nelle telecomunicazioni con il termine \emph{Visible Light Communication VLC} (ing. comunicazione attraverso luce visibile) si indica l'uso di luce visibile (luce con una frequenza di 400-800 THz/wavelength di 780-375 nm) come mezzo di trasmissione. VLC è un sottoinsieme delle tecnologie wireless di comunicazione ottica}
}

\newglossaryentry{itsg} {
    name=\glslink{its}{ITS},
    text=Intelligent Transportation System,
    sort=its,
    description={con il termine \emph{Intelligent Transportation System ITS} (ing. sistemi di trasporto intelligenti) si intende: l'integrazione delle conoscenze nel campo delle telecomunicazioni, elettronica, informatica - in breve, la “telematica” - con l'ingegneria dei trasporti, per la pianificazione, progettazione, esercizio, manutenzione e gestione dei sistemi di trasporto}
}

\newglossaryentry{iotg} {
    name=\glslink{iot}{IoT},
    text=Internet of Things,
    sort=iot,
    description={l'\emph{Internet of Things IoT} (ing. Internet delle cose (IdCa)) è un neologismo utilizzato nel mondo delle telecomunicazioni e dell'informatica che fa riferimento all'estensione di internet al mondo degli oggetti e dei luoghi concreti, che acquisiscono una propria identità digitale in modo da poter comunicare con altri oggetti nella rete stessa}
}


% TODO Termini da inserire
%  - OpenVLC?
%  - LED?