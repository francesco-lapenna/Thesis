\chapter{Conclusioni}
\label{cap:conclusioni}

% ================================================================= %
\section{Discussione}
In questa tesi è stato presentato un modello di autenticazione dei messaggi a livello fisico per OpenVLC. Sono stati approfonditi in dettaglio gli aspetti di progettazione ed implementazione di tale modello, il quale in fase di test, ha dimostrato di poter garantire maggior sicurezza alle comunicazioni \gls{vlc} non introducendo differenze significative nelle prestazioni della trasmissione.\\
È stata inoltre effettuata un'analisi dettagliata sul sistema OpenVLC, sulle sue componenti e sul suo funzionamento e condotti test esaustivi per valutare la \gls{vlc} in diverse condizioni di distanza, inclinazione e luce ambientale.

\section{Raggiungimento degli obiettivi}
L'obiettivo primario di questo progetto era progettare e implementare un modulo di autenticazione a livello fisico per la piattaforma OpenVLC, in grado di prevenire attacchi di replay e packet injection nei primi stadi della comunicazione. Come definito nel piano di lavoro, gli obiettivi obbligatori erano:
\begin{itemize}
  \item[O01] Configurare le schede BeagleBone Black per consentire la comunicazione tramite luce visibile;
  \item[O02] Studiare lo standard \gls{ieee} 802.15.7 per comprendere i requisiti tecnici e le specifiche;
  \item[O03] Progettare e implementare il modulo di autenticazione sulla piattaforma OpenVLC;
  \item[O04] Testare e validare il modulo sviluppato in ambiente buio;
  \item[O05] Documentare dettagliatamente il lavoro svolto e i risultati ottenuti.
\end{itemize}
Tutti questi punti sono stati soddisfatti: le schede sono state configurate con Debian 8 e driver OpenVLC\_VL\_IR 1.3, si è approfondito lo standard \gls{ieee} 802.15.7, e il modulo \texttt{PhyAuthP2P} è stato integrato nel driver OpenVLC e testato in diverse condizioni, confermando che l'introduzione del meccanismo di autenticazione permette di scartare pacchetti illegittimi non degradando le prestazioni.

L'obiettivo desiderabile D01 (Testare e validare il modulo sviluppato in ambiente con luce ambiente) è stato esplorato: i test non hanno evidenziato alcuna differenza rispetto al framework OpenVLC originale.

L'obbiettivo facoltativo F02 (Testare il sistema in condizioni più rumorose (con disturbo indotto)) è stato soddisfatto.
L'obbiettivo facoltativo F01 (Effettuare il fine tuning delle resistenze per ottimizzare la ricezione del segnale) non è stato esplorato.

\section{Conoscenze acquisite}
Nel corso di questo progetto ho avuto l'opportunità di approfondire competenze in:
\begin{itemize}
  \item Visible Light Communication e piattaforma OpenVLC, incluso l'uso di hardware e PRU su BeagleBone Black;
  \item Standard IEEE\,802.15.7, con focus su modulazione e codifica dei segnali;
  \item Sicurezza a livello fisico e crittografia, mediante studio di framework esistenti e l'implementazione di un meccanismo di generazione di \gls{otp};
  \item Progettazione software in C e Linux kernel-space;
  \item Metodologie di test e validazione, calcolo di metriche di valutazione del sistema \gls{vlc} e analisi di tali metriche.
\end{itemize}
Queste competenze rappresentano sia un ampliamento delle conoscenze teoriche sia uno sviluppo di abilità pratiche.

\section{Valutazione personale e sviluppi futuri}
L'esperienza di stage e tesi è stata altamente formativa. Le principali difficoltà incontrate sono state la configurazione iniziale dei cape OpenVLC e la gestione del sistema operativo Debian “jessie” in quanto obsoleto.

La soddisfazione maggiore deriva dall'aver realizzato un proof-of-concept funzionante, con impatto minimo sulle prestazioni e potenziale estendibilità. Come possibili sviluppi futuri di questo progetto si segnalano:
\begin{itemize}
  \item Integrazione di un Message Integrity Code per contrastare attacchi \gls{mitm};
  \item Valutazione di codici di correzione più potenti per ambienti ad alto rumore luminoso;
  \item Cifratura dei dati al fine di introdurre confidenzialità;
  \item Estensione del modulo a comunicazioni multipoint.
\end{itemize}