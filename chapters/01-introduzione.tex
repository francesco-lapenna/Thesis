\chapter{Introduzione}
\label{cap:introduzione}

% TODO: remove tutorial
Introduzione al contesto applicativo.\\

\noindent Esempio di utilizzo di un termine nel glossario \\
\gls{api}. \\

\noindent Esempio di citazione in linea \\
\cite{site:agile-manifesto}. \\

\noindent Esempio di citazione nel pie' di pagina \\
citazione\footcite{womak:lean-thinking} \\

% ----------------------------------------------------------------- %
\section{Contesto e motivazioni}
La \gls{vlc} è una tecnologia wireless che sfrutta la luce visibile, tipicamente LED, per trasmettere dati. Questo è reso possibile grazie alla modulazione della luce, ovvero a uno sfarfallio impercettibile all'occhio umano. \\
I principlali vantaggi che rendono questa tecnologia innovativa sono l'efficienza energetica, poiché \gls{vlc} sfrutta fonti di luce già designate all'illuminazione al fine di trasmettere dati, e la sicurezza, in quanto la luce visibile, a differenza delle onde radio, non può penetrare superfici opache. Il tutto garantendo al contempo una comunicazione ad alta velocità.\\
Gli ambiti di applicazione...

Introduci la lacuna di mancanza di autenticazione (spiega perché non c'è e perché è importante)
Spiega come abbiamo pensato di risolvere

% ----------------------------------------------------------------- %
\section{Obiettivi della tesi}

Introduzione all'idea e agli obbiettivi dello stage.

% ----------------------------------------------------------------- %
\section{Organizzazione del testo}

\begin{description}
    \item[{\hyperref[cap:background]{Il secondo capitolo}}] fornisce un background teoretico e tecnico necessario a comprendere il contesto e le tecnologie coinvolte nel progetto. 
    
    \item[{\hyperref[cap:analisi]{Il terzo capitolo}}] approfondisce la struttura dell'ambiente OpenVLC, le eventuali minacce alla sicurezza e le possibili contromisure.
    
    \item[{\hyperref[cap:progettazione]{Il quarto capitolo}}] descrive le scelte di progettazione ed i dettagli di implementazione del modulo di autenticazione sviluppato.
    
    \item[{\hyperref[cap:test]{Il quinto capitolo}}] descrive i test eseguiti e analizza le performance del sistema sviluppato.
        
    \item[{\hyperref[cap:conclusioni]{Nel sesto capitolo}}] si riassume i risultati ottenuti e i possibili sviluppi futuri.
\end{description}

\noindent \\Riguardo la stesura del testo, relativamente al documento sono state adottate le seguenti convenzioni tipografiche:
\begin{itemize}
	\item gli acronimi, le abbreviazioni e i termini ambigui o di uso non comune menzionati vengono definiti nel glossario, situato alla fine del presente documento;
	\item per la prima occorrenza dei termini riportati nel glossario viene utilizzata la seguente nomenclatura: \emph{parola}\glsfirstoccur;
	\item i termini in lingua straniera o facenti parti del gergo tecnico sono evidenziati con il carattere \emph{corsivo}.
\end{itemize}