\chapter{Introduzione}
\label{cap:introduzione}

% TODO: remove tutorial
Introduzione al contesto applicativo.\\

\noindent Esempio di utilizzo di un termine nel glossario \\
\gls{api}. \\

\noindent Esempio di citazione in linea \\
\cite{site:agile-manifesto}. \\

\noindent Esempio di citazione nel pie' di pagina \\
citazione\footcite{womak:lean-thinking} \\

% ================================================================= %
\section{Contesto e motivazioni}
La \gls{vlc} è una tecnologia wireless che sfrutta la luce visibile, tipicamente LED, per trasmettere dati. Questo è reso possibile grazie alla modulazione della luce, ovvero a uno sfarfallio impercettibile all'occhio umano.

I principali vantaggi che rendono questa tecnologia innovativa sono l'efficienza energetica, poiché \gls{vlc} sfrutta fonti di luce già designate all'illuminazione al fine di trasmettere dati, e la sicurezza, in quanto la luce visibile, a differenza delle onde radio, non può penetrare superfici opache. Il tutto garantendo al contempo una comunicazione ad alta velocità.

Tra i principali ambiti di applicazione vi sono: \gls{its}, permette ai veicoli di comunicare tra loro e con le infrastrutture circostanti in modo tale da ridurre il rischio di incidenti e ottimizzare il traffico; Smart Cities e Smart Homes, dove la comunicazione tramite luce visibile può essere sfruttata per la comunicazione sicura e veloce tra dispositivi \gls{iot}; Ambienti ospedalieri, nei quali la \gls{vlc} può offrire maggiore sicurezza riducendo eventuali rischi per la salute associati alle onde radio e minimizzando le interferenze con le apparecchiature mediche sensibili.

Sebbene la \gls{vlc}, come menzionato in precedenza, presenti di per sé caratteristiche che la rendano sicura, è comunque vulnerabile ad attacchi di tipo spoofing, in cui l'attaccante impersona un device legittimo, replay attacks, in cui l'attaccante intercetta e ritrasmette pacchetti legittimi, e denial of service (DoS), in cui l'attaccante inonda la rete con pacchetti per esaurire le risorse del sistema.

La \gls{vlc} è regolata dallo standard IEEE 802.15.7 che attualmente non definisce protocolli di sicurezza a livello fisico, in quanto assume che il canale di comunicazione sia sicuro e non accessibile a terzi e perciò delega la sicurezza ai livelli superiori. Tuttavia questa assunzione non è valida in tutti gli ambiti di applicazione, come ad esempio in ambienti aperti o in scenari di \gls{its}, dove i dispositivi possono essere facilmente accessibili da parte di attaccanti esterni.
In questi scenari, la sicurezza a livello fisico è decisamente vantaggiosa, in quanto permette al sistema di rilevare e scartare pacchetti sospetti prima che vengano elaborati dai livelli superiori, riducendo il consumo di risorse computazionali ed energetiche.

% ================================================================= %
\section{Obiettivi della tesi}

Questo progetto di tesi, si propone di sviluppare e integrare nel framework \cite{site:openvlc}, un modulo di autenticazione a livello fisico che permetta di prevenire alcuni tra i possibili rischi citati in precedenza, come spoofing, replay e DoS attacks, scartando pacchetti illeciti il prima possibile nella comunicazione e mantenendo al contempo le prestazioni del sistema.

% ================================================================= %
\section{Organizzazione del testo}

\begin{description}
    \item[{\hyperref[cap:background]{Il secondo capitolo}}] fornisce un background teoretico e tecnico necessario a comprendere il contesto e le tecnologie coinvolte nel progetto. 
    
    \item[{\hyperref[cap:analisi]{Il terzo capitolo}}] approfondisce la struttura dell'ambiente OpenVLC, le eventuali minacce alla sicurezza e le possibili contromisure.
    
    \item[{\hyperref[cap:progettazione]{Il quarto capitolo}}] descrive le scelte di progettazione ed i dettagli di implementazione del modulo di autenticazione sviluppato.
    
    \item[{\hyperref[cap:test]{Il quinto capitolo}}] descrive i test eseguiti e analizza le performance del sistema sviluppato.
        
    \item[{\hyperref[cap:conclusioni]{Nel sesto capitolo}}] si riassume i risultati ottenuti e i possibili sviluppi futuri.
\end{description}

\noindent \\Riguardo la stesura del testo, relativamente al documento sono state adottate le seguenti convenzioni tipografiche:
\begin{itemize}
	\item gli acronimi, le abbreviazioni e i termini ambigui o di uso non comune menzionati vengono definiti nel glossario, situato alla fine del presente documento;
	\item per la prima occorrenza dei termini riportati nel glossario viene utilizzata la seguente nomenclatura: \emph{parola}\glsfirstoccur;
	\item i termini in lingua straniera o facenti parti del gergo tecnico sono evidenziati con il carattere \emph{corsivo}.
\end{itemize}