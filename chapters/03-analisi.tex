\chapter{Analisi}
\label{cap:analisi}

\intro{In questo capitolo si approfondisce in dettaglio il funzionamento dell'ambiente OpenVLC, le minacce alla sicurezza a cui esso è soggetto e le possibili contromisure.}\\

% ================================================================= %
\section{Analisi del sistema}
come funziona l'ambiente di OpenVLC (descrizione ad alto livello del sistema)
\subsection{Configurazione e problematiche riscontrate}
descrivere la versione di OpenVLC usata e come è stato configurato il sistema
descrivi anche le problematiche riscontrate durante la configurazione e come le hai risolte
di che per com'è scritto il codice ci sono gravi problemi di sincronizzazione dato che la ricezione del preambolo è commentata
installazione os e script fatti per automatizzare la configurazione

% ================================================================= %
\section{Vulnerabilità}
cosa assumi sull'attaccante: posizione buona per intercettare il fascio, rubare i pacchet-
ti, rubare le password, modificare pacchetti, distruggere pacchetti a caso o mandare
pacchetti a caso

% ================================================================= %
\section{Soluzione proposta}
spiega come funziona l'autenticazione a livello fisico.