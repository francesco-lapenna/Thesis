\chapter{Progettazione ed implementazione}
\label{cap:progettazione}

\intro{In questo capitolo si descrive il workflow del modulo sviluppato, le scelte di progettazione ed i dettagli di implementazione del modulo di autenticazione sviluppato.}\\

% ================================================================= %
\section{Progettazione}

\paragraph{Premessa e Workflow.}
Essendo una prima versione sperimentale, questo metodo di autenticazione, come del resto OpenVLC, è stato progettato per una comunicazione \gls{p2p}\glsfirstoccur e assume che i due nodi abbiano già stabilito una connessione.
Il workflow consiste in due passi: il trasmettitore, in fase di invio dei dati, calcola una \gls{otp} e la incorpora in una posizione nota nei segnali fisici del pacchetto; il ricevitore, in fase di ricezione, estrae la \gls{otp} dai segnali fisici del pacchetto e ne verifica la validità.\\
I pacchetti che non presentano una valida \gls{otp} vengono scartati, altrimenti possono procedere ad ulteriori elaborazioni da parte di OpenVLC.


\subsection{Generazione e verifica dell'OTP}
Una \gls{otp} è un codice di autenticazione temporaneo, specifico per dispositivo in quanto involve una chiave segreta, valido per una sola operazione o comunque per un breve periodo di tempo.

\paragraph{Generazione.}
\paragraph{Lunghezza.}
\paragraph{HMAC.}
parla di \gls{hmac} 
\paragraph{Time-step T.}
\paragraph{Chiave Segreta K.}
\paragraph{Sequence Number SN.}
\paragraph{Verifica.}


\subsection{Codifica e decodifica dell'OTP}

\paragraph{Embedding e Trasmissione.}
parla dello \gls{xor}
\paragraph{Estrazione.}



% ================================================================= %
\section{Implementazione}
% spiega tutto quello che hai fatto testbed codice modificato (motivazione scelte)






Tutte le modifiche apportate sono tracciate nella repository del progetto nel branch PhyAuthP2P: \cite{site:openvlc-pa-github}.