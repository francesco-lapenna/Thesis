\chapter{Background}
\label{cap:background}

\intro{In questo capitolo si introducono brevemente i concetti fondamentali necessari alla comprensione del contenuto della tesi.}

% ----------------------------------------------------------------- %
\section{Visible Light Communication}
\subsection{Introduzione}
Nelle telecomunicazioni, la \gls{vlc}, talvolta indicata anche come "LiFi", prevede l'uso della luce visibile (cioè la luce avente frequenza di 400-800 THz e lunghezza d'onda di 780-375 nm) come mezzo di trasmissione. La \gls{vlc} è un sottoinsieme delle tecnologie di comunicazione ottica wireless.

Questa tecnologia usa comuni lampade flourescenti e LED per trasmettere segnali da 10 kbit/s fino a 500 Mbit/s su corte distanze.

Generalmente il segnale luminoso viene ricevuto da un dispositivo elettronico dotato di fotodiodo. Tuttavia in alcuni casi è sufficiente una fotocamera digitale, la quale, essendo un insieme di fotodiodi, potrebbe addirittura essere preferibile in quanto è in grado di ricevere segnali luminosi a diverse frequenze, permettendo così la trasmissione di più canali contemporaneamente.

\subsection{Vantaggi e applicazioni}
Una delle principali caratteristiche della \gls{vlc}, nonché la principale ragione della sua sicurezza, è l'incapacità della luce visibile di attraversare superfici opache. Se la comunicazione avviene in un ambiente chiuso, tale caratteristica permette di confinare la comunicazione a quell'ambiente. Di conseguenza, costringe un potenziale attaccante che vuole intercettare la comunicazione ad avere accesso fisico a tale ambiente.

Un'altra caratteristica della \gls{vlc} è la possibilità di essere integrata a fonti di luce pre-esistenti, e di conseguenza essere usata al duplice scopo di formire luce e trasmettere dati. Questa caratteristica apre le porte all'\gls{ubicomp}\glsfirstoccur, in quanto i dispositivi che emettono luce (come lampade, LED, schermi, semafori, ecc.) sono già presenti in abbondanza in ogni ambiente;

Un ulteriore ambito di applicazione promettente è l'\gls{ips}\glsfirstoccur. Analogamente al GPS, permette di localizzare un dispositivo, ma è in grado di operare in spazi chiusi dove il GPS non è disponibile.

% ----------------------------------------------------------------- %
\section{Standard IEEE 802.15.7}
L'\gls{ieee}\glsfirstoccur è un'organizzazione internazionale di ingegneri che si occupa di standardizzare le tecnologie di comunicazione.

Lo standard che regola la \gls{vlc} è l'IEEE 802.15.7\footcite{ieee802157}. Questo standard definisce le specifiche per la comunicazione ottica wireless a corto raggio. Si concentra, in particolare, sul livello fisico e sul livello di accesso al mezzo.

Questo standard è in grado di fornire velocità di trasmissione dati sufficienti a supportare servizi multimediali audio e video e considera anche la mobilità del collegamento ottico, la compatibilità con varie infrastrutture luminose, le limitazioni dovute al rumore e alle interferenze da fonti come la luce ambientale e un sottolivello MAC che soddisfa le esigenze uniche dei collegamenti visibili e delle altre lunghezze d'onda della luce.

Tuttavia, non definisce protocolli di sicurezza a livello fisico, in quanto assume che il canale di comunicazione sia sicuro e non accessibile a terzi. Perciò delega la sicurezza ai livelli superiori.

% ----------------------------------------------------------------- %
\section{Modello OSI}

L'\gls{osi}\glsfirstoccur è lo standard di riferimento per le reti di calcolatori. Sviluppato dalla International Organization for Standardization (ISO), definisce un'architettura sviluppata su sette livelli: Physical, Data Link, Network, Transport, Session, Presentation, ed Application.\\
I livelli più inerenti a quanto trattato in questa tesi sono i primi due: \gls{phy}\glsfirstoccur e \gls{mac}\glsfirstoccur il quale è un sottolivello del livello Data Link.

Il \gls{phy}, è il primo livello del modello OSI. Si occupa della conversione di dati in segnali elettrici, radio o ottici e della trasmissione e ricezione di tali dati grezzi su un canale di comunicazione fisico.

il \gls{mac}, è il livello che controlla l'hardware responsabile dell'interazione con il mezzo di trasmissione. Il livello \gls{mac} è responsabile della gestione dell'accesso al mezzo di trasmissione, della sincronizzazione dei dispositivi e della gestione degli errori. Insieme al logical link control layer (LLC) costituisce il data link layer del modello OSI.

% ----------------------------------------------------------------- %
\section{OpenVLC e BeagleBone}
OpenVLC\footcite{site:openvlc} è una piattaforma open source per la comunicazione ottica wireless, sviluppata dal Pervasive Wireless Systems group del Dr. Giustiniano all'IMDEA Networks Institute (Madrid, Spagna).
OpenVLC è progettata per essere flessibile e low-cost, permettendo, nell'ultima versione, la trasmissione di dati ad una velocità di 400 kb/s e ad una distanza di quasi 20 metri.

OpenVLC è progetatto per funzionare su BeagleBone\footcite{site:beaglebone}Black, una piattaforma hardware open source low-cost per sviluppatori, dotata di un processore ARM Cortex-A8 e di una serie di interfacce di comunicazione, tra cui USB, Ethernet e GPIO. BeagleBone è particolarmente adatta per applicazioni embedded e IoT, grazie alla sua flessibilità e alle sue capacità di elaborazione.

% ----------------------------------------------------------------- %
\section{PhyAuth USENIX}
Un progetto di ricerca recente, analogo a quello discusso in questa tesi, e da cui è stata presa ispirazione, è PhyAuth\footcite{phyauth2023}.\\
PhyAuth è un framework di autenticazione hop-by-hop dei messaggi a livello fisico; lo scopo è quello di difendere le reti ZigBee da attacchi packet-injection.

L'idea alla base di PhyAuth è quella di inglobare, nei segnali fisici di ogni messaggio trasmesso, una \gls{otp}\glsfirstoccur calcolata sulla base di una chiave segreta device-specifica e una funzione crittografica di hash.\\
La validità di tale chiave viene verificata dal dispositivo ricevente, il quale, se la verifica non va a buon fine, scarta l'intero pacchetto.

Questo framework permette di rilevare accuratamente pacchetti sospetti con poco impatto sulle prestazioni del sistema, e con poche modifiche al protocollo ZigBee, assicurando inoltre compatibilità con i dispositivi su cui non è installato.
