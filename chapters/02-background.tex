\chapter{Background}
\label{cap:background}

\intro{In questo capitolo si introducono brevemente i concetti fondamentali necessari alla comprensione del contenuto della tesi.}\\

% ----------------------------------------------------------------- %
\section{Visible Light Communication}
\subsection{Introduzione}
Nelle telecomunicazioni, la \gls{vlc} prevede l'uso della luce visibile (cioè la luce avente frequenza di 400-800 THz e lunghezza d'onda di 780-375 nm) come mezzo di trasmissione. La \gls{vlc} è un sottoinsieme delle tecnologie di comunicazione ottica wireless.\\
Questa tecnologia usa comuni lampade flourescenti e LED per trasmettere segnali da 10 kbit/s fino a 500 Mbit/s su corte distanze.\\
Generalmente il segnale luminoso viene ricevuto da un dispositivo elettronico dotato di fotodiodo. Tuttavia in alcuni casi è sufficiente una fotocamera digitale, la quale, essendo un insieme di fotodiodi, potrebbe addirittura essere preferibile in quanto è in grado di ricevere segnali luminosi a diverse frequenze, permettendo così la trasmissione di più canali contemporaneamente.\\

\subsection{Vantaggi e applicazioni}
Una delle principali caratteristiche della \gls{vlc}, nonché la principale ragione della sua sicurezza, è l'incapacità della luce visibile di attraversare superfici opache. Se la comunicazione avviene in un ambiente chiuso, tale caratteristica permette di confinare la comunicazione a quell'ambiente. Di conseguenza, costringe un potenziale attaccante che vuole intercettare la comunicazione ad avere accesso fisico a tale ambiente.\\
Un'altra caratteristica della \gls{vlc} è la possibilità di essere integrata a fonti di luce pre-esistenti, e di conseguenza essere usata al duplice scopo di formire luce e trasmettere dati. Questa caratteristica apre le porte all'\gls{ubicomp}\glsfirstoccur, in quanto i dispositivi che emettono luce (come lampade, LED, schermi, semafori, ecc.) sono già presenti in abbondanza in ogni ambiente;\\
Un ulteriore ambito di applicazione promettente è l'\gls{ips}\glsfirstoccur. Analogamente al GPS, permette di localizzare un dispositivo, ma è in grado di operare in spazi chiusi dove il GPS non è disponibile.\\

% ----------------------------------------------------------------- %
\section{Standard IEEE 802.15.7}

% ----------------------------------------------------------------- %
\section{OSI model???}

% ----------------------------------------------------------------- %
\section{OpenVLC e BeagleBone}

% ----------------------------------------------------------------- %
\section{Autenticazione e crittografia}
